\documentclass[]{article}

%opening

\title{Identifying macro-moths with micro-features}
\author{Paul J. Palmer}

\begin{document}

\maketitle

\begin{abstract}

This article explores the use of external microscopic features to support the identification (ID) of  macro-lepidoptera. The usual process of identifying macro-moths often focusses on the wings, which are often large and distinctively patterned. Matching an unknown specimen to reference pictures is often the identification method employed given the lack of systematic keys for Lepidoptera in general and coupled with experience, can give good results. The article uses a problematic identification as an example of how microscopic features may be used to narrow the field of candidate taxa to arrive at a specific taxon. 

\end{abstract}

\section*{A difficult ID}


\end{document}
